\documentclass[a4paper,10pt]{article}
\usepackage{pslatex}
\usepackage[utf8]{inputenc}
\usepackage{multicol}
\usepackage{multirow}
\usepackage{graphicx}
\usepackage[table]{xcolor}
\usepackage[left=2cm,top=2cm,right=2cm,bottom=2cm]{geometry}
\usepackage[brazil]{babel}

%+---------------------------------------------+
%|  template criado por @maerli.pereira        |
%|  09 de março de 2018                        |
%|  base do template criado por Paulo Santiago |
%+---------------------------------------------+

%nome do curso
\newcommand{\curso}{Licenciatura em Física}

%nome da disciplina
\newcommand{\disciplina}{Eletricidade e Magnetismo I – 2017/2}

%nome do professor
\newcommand{\professor}{Prof. Maerli Pereira}

%nome da instituição
\newcommand{\instituicao}{Instituto Federal de Educação, Ciência e Tecnologia do Ceará – Campus Acaraú}

%etapa
\newcommand{\etapa}{N1}

%prova
\newcommand{\prova}{1}

%configuração
%criar comando \quest para formatação dos itens
\usepackage{calc}
\usepackage{ifthen}
\newcounter{mine}
\setcounter{mine}{1}
\newcommand{\quest}{\ifthenelse{ \equal{\themine}{1} }{
}{
  \vspace{0.5cm}
}  \noindent \textbf{\themine. } \setcounter{mine}{\themine + 1}}
%fim configuração

\pagestyle{empty}% sem paginação

\begin{document}
%\includegraphics[width=4cm,height=3cm]{c}
\begin{table}
   \hspace{-0.4cm}
   \begin{tabular}{ll}
        \multirow{4}{*}{ \includegraphics[width=1.4cm,height=1.61cm]{img/if.png}} 
        &\instituicao\\
        &\noindent\curso\\
        &\disciplina  - \professor\\
        & Prova \prova, Etapa \etapa \\
    \end{tabular}
\end{table}

\vspace{0.3cm}
\hspace{-0.5cm}\fbox{Aluno(a):\hspace{11cm} Data:\hspace{3.5cm}}

\begin{multicols*}{2} 
    \setlength{\columnseprule}{0.2pt}
    \setlength{\columnsep}{2cm}
    
    \quest Duas esferas muito pequenas, de 8,55 g, estão a uma distância de 15,0 cm de um centro a outro e são carregadas adicionando-se um número igual de elétrons a cada uma delas. Desconsiderando-se todas as demais forças, quantos elétrons teriam que ser adicionados a cada esfera para que ambas acelerem a 25,0g quando forem libertadas? Para que lado elas vão acelerar? [2,0 pontos]
    
    \quest Um elétron é projetado com velocidade inicial $v_0=1,60×10^6$  m/s para dentro do campo elétrico uniforme entre placas paralelas indicadas na Figura 21.38. Suponha que o campo seja uniforme e orientado verticalmente para baixo e considere igual a zero o campo elétrico fora das placas. O elétron entra no campo em um ponto intermediário entre as placas. \textbf{(a)} Sabendo que o elétron tangencia a placa superior quando ele sai do campo, calcule o módulo do campo elétrico. \textbf{(b)} Suponha que na Figura 21.38 o elétron seja substituído por um próton com a mesma velocidade $v_0$. O próton colide com uma das placas? Se o próton não colide com nenhuma placa, qual deve ser o módulo, a direção e o sentido do seu deslocamento vertical quando ele sai da região entre as placas? \textbf{(c)} Compare as trajetórias seguidas pelo elétron e pelo próton e explique as diferenças. \textbf{(d)} Analise se é razoável desprezar os efeitos da gravidade para cada partícula. [1,0 ponto cada item]
    	
    \quest Uma carga puntiforme igual a +2,0 nC está na origem e uma segunda carga puntiforme igual a -5,0 nC encontra-se sobre o eixo Ox, no ponto x=0,800 m. (a) Determine o módulo, a direção e o sentido do campo elétrico nos seguintes pontos sobre o eixo Ox: (i) x=0,200 m; (ii) x=1,20 m; (iii) x=-0,200 m. (b) Calcule a força elétrica resultante que essas cargas exerceriam sobre um elétron situado em cada um dos pontos mencionados no item (a). [1,5 pontos cada item]
    	
    \quest A Figura 21.41 mostra algumas linhas de campo elétrico produzidas por três cargas puntiformes localizadas ao longo de um eixo vertical. Todas as três cargas possuem o mesmo módulo. (a) Quais são os sinais de cada uma das três cargas? Explique seu racio-cínio. (b) Em que ponto(s) o módulo do campo elétrico atinge seu valor mínimo? Explique como os campos produzidos pelas cargas individuais se combinam para fornecer o campo elétrico resultante nesse(s) ponto(s). [1,5 pontos cada item]
    	
    		
    \quest Um elétron é projetado com velocidade inicial $v_0=1,60×10^6$  m/s para dentro do campo elétrico uniforme entre placas paralelas indicadas na Figura 21.38. Suponha que o campo seja uniforme e orientado verticalmente para baixo e considere igual a zero o campo elétrico fora das placas. O elétron entra no campo em um ponto intermediário entre as placas. (a) Sabendo que o elétron tangencia a placa superior quando ele sai do campo, calcule o módulo do campo elétrico. (b) Suponha que na Figura 21.38 o elétron seja substituído por um próton com a mesma velocidade $v_0$. O próton colide com uma das placas? Se o próton não colide com nenhuma placa, qual deve ser o módulo, a direção e o sentido do seu deslocamento vertical quando ele sai da região entre as placas? (c) Compare as trajetórias seguidas pelo elétron e pelo próton e explique as diferenças. (d) Analise se é razoável desprezar os efeitos da gravidade para cada partícula. [1,0 ponto cada item]
    
    \quest Um elétron é projetado com velocidade inicial $v_0=1,60×10^6$  m/s para dentro do campo elétrico uniforme entre placas paralelas indicadas na Figura 21.38. Suponha que o campo seja uniforme e orientado verticalmente para baixo e considere igual a zero o campo elétrico fora das placas. O elétron entra no campo em um ponto intermediário entre as placas. (a) Sabendo que o elétron tangencia a placa superior quando ele sai do campo, calcule o módulo do campo elétrico. (b) Suponha que na Figura 21.38 o elétron seja substituído por um próton com a mesma velocidade $v_0$. O próton colide com uma das placas? Se o próton não colide com nenhuma placa, qual deve ser o módulo, a direção e o sentido do seu deslocamento vertical quando ele sai da região entre as placas? (c) Compare as trajetórias seguidas pelo elétron e pelo próton e explique as diferenças. (d) Analise se é razoável desprezar os efeitos da gravidade para cada partícula. [1,0 ponto cada item]
    
    \quest Um elétron é projetado com velocidade inicial $v_0=1,60×10^6$  m/s para dentro do campo elétrico uniforme entre placas paralelas indicadas na Figura 21.38. Suponha que o campo seja uniforme e orientado verticalmente para baixo e considere igual a zero o campo elétrico fora das placas. O elétron entra no campo em um ponto intermediário entre as placas. (a) Sabendo que o elétron tangencia a placa superior quando ele sai do campo, calcule o módulo do campo elétrico. (b) Suponha que na Figura 21.38 o elétron seja substituído por um próton com a mesma velocidade $v_0$. O próton colide com uma das placas? Se o próton não colide com nenhuma placa, qual deve ser o módulo, a direção e o sentido do seu deslocamento vertical quando ele sai da região entre as placas? (c) Compare as trajetórias seguidas pelo elétron e pelo próton e explique as diferenças. (d) Analise se é razoável desprezar os efeitos da gravidade para cada partícula. [1,0 ponto cada item]
    
     \quest Um elétron é projetado com velocidade inicial $v_0=1,60×10^6$  m/s para dentro do campo elétrico uniforme entre placas paralelas indicadas na Figura 21.38. Suponha que o campo seja uniforme e orientado verticalmente para baixo e considere igual a zero o campo elétrico fora das placas. O elétron entra no campo em um ponto intermediário entre as placas. (a) Sabendo que o elétron tangencia a placa superior quando ele sai do campo, calcule o módulo do campo elétrico. (b) Suponha que na Figura 21.38 o elétron seja substituído por um próton com a mesma velocidade $v_0$. O próton colide com uma das placas? Se o próton não colide com nenhuma placa, qual deve ser o módulo, a direção e o sentido do seu deslocamento vertical quando ele sai da região entre as placas? (c) Compare as trajetórias seguidas pelo elétron e pelo próton e explique as diferenças. (d) Analise se é razoável desprezar os efeitos da gravidade para cada partícula. [1,0 ponto cada item]
    
    
    \vspace{2cm}
    
    \scriptsize{
    \noindent
    \begin{tabular}{| m{3.9cm} p{3.5cm} |}
         \hline
        \multicolumn{2}{| {\centeringm{5cm}}|}{Dados úteis aos enunciados \cellcolor[gray]{0.5}} \\
        \hline
          Carga elementar & $e \approx 1,602 \times 10^{-19}  C$  \\
          \hline
          Massa do elétron & $m_e=9,11 \times 10^{-31}  kg$\\
          \hline
          Massa do próton & $m_p=1,67 \times 10^{-27}  kg$ \\
          \hline
          Constante eletrostática & $k \approx 8,988 \times 10^9  Nm^2/C^2$ \\
          \hline
          Aceleração da gravidade	& $g \approx 9,807 m/s^2$ \\
          \hline
          Intensidade da força elétrica (lei de Coulomb) entre cargas puntiformes &	$F_e=k\frac{|q_1 q_2| }{r^2}$ \\
          \hline
          Força e campo elétricos & $\vec{F_e} =q\vec{E}$ \\
          \hline
    \end{tabular}
    \vspace{0.4cm}
    
     \noindent
     \begin{tabular}{|p{7.8cm}|}
        \hline
        \cellcolor[gray]{0.8}
        Observações: a pontuação total da prova chega a 12,0 pontos, enquanto a nota máxima é 10,0. Responda tudo o que puder, a pontuação que exceder a nota máxima será desconsiderada.
        \\
        \hline
    \end{tabular}
    }
    \vspace{0.2cm}
    \begin{center}
        “Não existem métodos fáceis para resolver problemas difíceis” - René Descartes\\
        \vspace{0.2cm}
        \textit{Boa prova!}
    \end{center}
\end{multicols*}
 
\end{document}